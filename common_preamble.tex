% Latex preamble that can be included in all subdocuments of a compound lyx/tex
% file.  This allows each subfile to be compiled on its own for quick checks,
% while the master file can also be built for final distribution with full
% cross-references, bibliography, etc.
%-----------------------------------------------------------------------------
% Special-purpose color definitions (dark enough to print OK in black and white)
\usepackage{color}

% A few colors to replace the defaults for certain link types
\definecolor{orange}{cmyk}{0,0.4,0.8,0.2}
\definecolor{darkorange}{rgb}{.71,0.21,0.01}
\definecolor{darkgreen}{rgb}{.12,.54,.11}

%-----------------------------------------------------------------------------
% The hyperref package gives us a pdf with properly built
% internal navigation ('pdf bookmarks' for the table of contents,
% internal cross-reference links, web links for URLs, etc.)
\usepackage{hyperref}
\hypersetup{pdftex,  % needed for pdflatex
  breaklinks=true,  % so long urls are correctly broken across lines
  colorlinks=true,
  urlcolor=blue,
  linkcolor=darkorange,
  citecolor=darkgreen,
  }

%-----------------------------------------------------------------------------
% This helps prevent overly long lines that stretch beyond the margins
\sloppy

%-----------------------------------------------------------------------------
% Use and configure listings package for nicely formatted code
\usepackage{listings}
\lstset{
  language=Python,
  basicstyle=\small\ttfamily,
  commentstyle=\ttfamily\color{blue},
  stringstyle=\ttfamily\color{darkorange},
  showstringspaces=false,
  breaklines=true,
  postbreak = \space\dots
}

%-----------------------------------------------------------------------------
%
% Commands for annotating the docs with fixme and inter-author notes.  See
% below for how to disable these.
%
% Define a \fixme command to mark visually things needing fixing in the draft,
% as well as similar commands for each author to leave initialed special
% comments in the document.
% For final printing or to simply disable these bright warnings, copy
% (there's a target macros_off' in the makefile that does this) the file
% macros_off.tex to macros.tex

\newcommand{\fix}[1] { \textcolor{red} {
{\fbox{ {\bf FIX} \ensuremath{\blacktriangleright }} {\bf #1}
\fbox{\ensuremath{\blacktriangleleft} } } } }

% And similarly, one (less jarring, with fewer symbols and no boldface) command
% for each one of us to leave comments in the main text.
\newcommand{\fperez}[1] { \textcolor{blue} {
\ensuremath{\blacklozenge} {\bf fperez:}  {#1}
\ensuremath{\blacklozenge} } }

\newcommand{\bgranger}[1] { \textcolor{darkgreen} {
\ensuremath{\bullet} {\bf bgranger:}  {#1}
\ensuremath{\bullet} } }

\newcommand{\mbrett}[1] { \textcolor{darkorange} {
\ensuremath{\blacksquare} {\bf mbrett:}  {#1}
\ensuremath{\blacksquare} } }

\newcommand{\jb}[1] { \textcolor{darkgreen} {
\ensuremath{\bigstar} {\bf jb:}  {#1}
\ensuremath{\bigstar} } }

% Input a file that can optionally disable these special macros.  By writing
% this file out, we can de-activate these macros with one call to produce final
% distribution-ready versions of the document.  Use the makefile target 'make
% macros_off' for this purpose.
%\input{macros}

% Label figures consequetively, rather than per section
% \usepackage{caption}
% \captionsetup{%
%   figurewithin=none,
%   tablewithin=none
% }
