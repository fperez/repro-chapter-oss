\documentclass[ChapterTOCs,krantz2]{krantz} % Use krantz2 for 7" x 10" trim size
\usepackage{graphicx}
\usepackage{subfigure}

%-----------------------------------------------------------------------------
% The hyperref package gives us a pdf with properly built
% internal navigation ('pdf bookmarks' for the table of contents,
% internal cross-reference links, web links for URLs, etc.)
%\usepackage{hyperref}

\usepackage{url}

%% Define a new 'leo' style for the package that will use a smaller font.
\makeatletter
\def\url@leostyle{%
  \@ifundefined{selectfont}{\def\UrlFont{\sf}}{\def\UrlFont{\small\ttfamily}}}
\makeatother
%% Now actually use the newly defined style.
\urlstyle{leo}

\begin{document}

\title{Dummy title}
\author{Dummy author}
\chapter*{Dummy chapter needed for the \textbackslash chapterauthor command to work later}

\mainmatter

\chapterauthor{Fernando P\'{e}rez}{Henry H. Wheeler Jr. Brain Imaging Center\\
Helen Wills Neuroscience Institute\\
University of California, Berkeley}
\chapterauthor{K. Jarrod Millman}{Division of Biostatistics\\
School of Public Health\\
University of California, Berkeley}


\chapter{Open source software and scientific research}

As active members of both the scientific research and the open-source
software development communities, we have observed that the latter
often lives up better than the former to our ideals of scientific
openness and reproducibility.  We explore the reasons behind this,
and argue that these problems are particularly acute in computational
domains where they should be in fact less prevalent.   We discuss
how we can draw specific lessons from the open source community both
in terms of technical approaches and of changing the structure of
incentives, to make progress towards a more solid base for reproducible
computational research.

\section{Introduction}\label{intro}
\begin{itemize}
\item Levels of reproducibility: replication, validation, reproduction,
new construction.
\item Conditions: clarity, transparency and traceability, predictability
(which requires automation), communicability
\end{itemize}

\subsection{Crisis in computationally-driven science}

Computing: part of the DNA of science

Much more than ``the third branch'' of science
\begin{itemize}
\item An avalanche of experimental quantitative data

\begin{itemize}
\item Biology, genetics, neuroscience, astronomy, climate modeling...
\end{itemize}
\item All scientists must now do real computing
\item ``Big Data'', ``Cloud computing'', etc: lots of buzzwords...

\begin{itemize}
\item They will \textbf{NOT} automatically produce good science
\end{itemize}
\item Good computing is now a necessary (though not
sufficient!) condition for good science.
\end{itemize}
A crisis of credibility and real issues
\begin{itemize}
\item <+>\textbf{The Duke clinical trials }\textbf{scandal}\textbf{
}- Potti/Nevin

\begin{itemize}
\item A compounding of (common and otherwise) data analysis errors.
\item No materials allowing validation/reproduction of results.
\item \textbf{Patients were harmed}.
\item Lawsuits, resignations.
\item Major policy reviews and changes: NCI, IOM, ...
%\item More: see ef{http://bioinformatics.mdanderson.org/Supplements/ReproRsch-All/Modified/StarterSet/index.html}{K. Baggerly's "starter set" page}.
\end{itemize}
\item The Duke situation is more common than we'd like
to believe!

\begin{itemize}
\item Begley \& Ellis, Nature, 3/28/12: {\emph Drug development:
Raise standards for preclinical cancer research.}
\item 47 out of 53 ``landmark papers'' could not be replicated.
\end{itemize}
\item Nature, Feb 2012, Ince et al: {\emph The case
for open computer programs}

\begin{itemize}
\item The scientific community places more faith in computation
than is justified
\item anything less than the release of actual source code is an
indefensible approach for any scientific results that depend on computation
\end{itemize}
\item Retraction rates are going up (find citation)
\end{itemize}

\subsection{A contrast in cultures }

Open source software development uses public fora for most discussion
and systems for sharing code and data that are, in practice, powerful
provenance tracking systems. There is a strong culture of public disclosure,
tracking and fixing of bugs, and development often includes exhaustive
validation tests that are executed automatically whenever changes
are made to the software and whose output is publicly available on
the internet. This helps with early detection of problems, mitigates
their recurrence, and ensures that the state and quality of the software
is a known quantity under a wide variety of situations (operating
systems, inputs, parameter ranges, etc). Additionally, the same systems
that are used for sharing the code track the authorship of contributions.
All of this ensures that open collaboration does not dilute the merit
or recognition of any individual developer, and allows for a meritocracy
of contributors to develop while enabling highly effective collaboration.

In sharp contrast, the incentives in computational research are strongly
biased towards the rapid publication of papers without any realistic
requirement of validation. The outcome is that results from publications
in computationally-based research (applied to any specific field of
inquiry) are often, in practice, impossible to reproduce. Sometimes
this is due to the code not being available at all in the first place.
Authors often do make codes available --thus fulfilling a token requirement
of disclosure-- but in such a state that this disclosure is not a
practical solution to the reproducibility problem. A static archive
of source code that has never been tested in a computer or operating
system outside of the author's, never been audited by external eyes
and with no automatic testing built into it, is highly unlikely to
work reliably when used in a completely new environment.


\subsection{The realities of transplanting approaches }

Notwithstanding the above, there are real issues with attempting to
na�vely transplant the practices of open source development verbatim
to computational research. The open source model ends up being one
where, in practice, the copyright and authorship of any large collaborative
project is spread amongst many authors, possibly thousands. While
the source control tools in use do allow for a relatively precise
provenance analysis to be performed if desired, this is rarely done
and its success is contingent on the community having followed certain
steps rigorously to ensure that attribution was correctly recorded
during development.

This is not a major issue in open source, as the rewards mechanisms
tend to be more informal and based on the overall recognition of any
one contributor in the community. Sometimes people contribute to open
source projects as part of their official work responsibilities, and
in that case a company can enact whatever policies it deems necessary;
often contributions are made by volunteers for whom an acknowledgment
in the project's credits is sufficient recognition.

In contrast, the academic world overwhelmingly weighs the authorship
of scholarly articles and conference proceedings as the main driver
of all forms of professional advancement and reward. In this system,
the pecking order of authorship matters enormously (with the many
unpleasant consequences familiar to all of us), and so does the total
number of authors in a publication. While in certain communities papers
with thousands of authors do exist (experimental high-energy physics
being the classic example), most scientists need the prominent visibility
they can achieve in a short author list. This dilution of authorship
that can result from a largely open collaborative development model
is an important issue that must be addressed by any proposal we present.

Furthermore, the notion of a fully open development model typical
of open source projects is at sharp odds with another aspect of the
scientific publication and reward system: the ``first
to publish'' race. Most scientists would, understandably,
be very leery of exposing their projects to an openly accessible website
when in their embryonic stages. The fear of being scooped by others
is very real, and again we must properly address it as we consider
how to apply the lessons of open source development to the scientific
context.


\subsection{The limits of scientific computational reproducibility }

As we seek to learn how the open source praxis can inform our scientific
work, we must recognize that the ideal of scientific reproducibility
is by necessity a reality of shades. We can see a gradation that goes
from a pure mathematical result whose proof should be accessible to
any person skilled in the necessary specialty, to one-of-a-kind experiments
such as the Large Hadron Collider or the Hubble Space Telescope, that
can't be reproduced in any realistic sense. At each point in this
spectrum, however, we can always find ways to improve our confidence
in the results: whether we re-analyze the same unique datasets with
independently developed packages run by separate groups or we re-acquire
partial samplings of critical data multiple times, we should never
completely renounce the ideals of reproducibility because of practical
difficulties.

Similarly, in computational research we also have certain areas where
complete reproducibility is more challenging than others. Some projects
require computations carried on the largest supercomputers on the
planet, and these are very expensive resources that can't be arbitrarily
allocated for repeated executions of the same problem. Others may
require access to enormous datasets that can't easily be transferred
to the desktop of any researcher wishing to re-execute an analysis.
But again, alternatives exist: it should be possible to validate scaled
versions of the largest problems run independently, against scaled
specimens created on the supercomputers for this very purpose, and
sub-sampled datasets can be used to collect at least validation statistics
that may be informative of the trust we place on the published analysis.

 
\subsection{Some ideas moving forward }

Ultimately, while it is true that there are real issues with applying
the ideals of computational reproducibility from open source software
development to computational research, we can and must do better.
We sketch here some ideas on concrete lessons we can learn from software
development in this direction:
\begin{itemize}
\item Pervasive version control: research codes should be developed, while
still in-house, \emph{always} using version control systems that track
the actual history of everyone's contributions.
\item Journals should mandate that upon paper \emph{approval} (but before
actual publication and with said publication being conditioned on
the author meeting this last condition), authors must expose their
version control system to the public, and that the publicly available
version can faithfully reproduce (within the limitations discussed
above) the published results. This public version then becomes available
for the scientific community not only for download, but also as a
starting point for further contribution and development.
\item By using a distributed version control system, authors can continue
to maintain a private branch where new work (say leading to a new
publication) is conducted while tracking the public development. This
will enable them to maintain exclusive access to their new work until
it is published, while continuing to develop the openly accessible
code with the rest of the scientific community. Once the code is published,
since it was developed using the same version control machinery of
the public branch, the new contributions can be seamlessly merged
with the public version and their entire provenance (including information
such as time of invention and individual credit within the lab) becomes
available for inspection.
\end{itemize}
In summary, we think that a few simple lessons can be learned from
the practices of the open source world which, if carefully assimilated,
can lead to significant improvements in the state of reproducible
computational research. 




%\begin{center}
%{\large The }\textcolor{blue}{\large rigor}{\large , }\textcolor{blue}{\large openness}{\large ,
%culture of }\textcolor{blue}{\large validation}{\large , }\textcolor{blue}{\large collaboration}{\large{}
%and other aspects of science }\textbf{\textcolor{blue}{\large must}}{\large{}
%also become part of scientific computing.}
%\par\end{center}{\large \par}

\section{Lessons from Open Source}

\subsection{A history of two cultures}


\subsubsection{The ideals of science and the reality of today's praxis}

Royal society motto, religion vs science, central authority vs the
ability of individuals to verify, the triumph of reason.


\subsubsection{The early days of computing}

The early days of computing \emph{were open}: von Neumann's reports
(SIAM article, SIAM review \textbf{53} (4) 2011, pp607-682. A history
of how open early computing was. 


\subsubsection{The software crisis}

Dijkstra's quote, 1972 ACM.


\subsubsection{The world of OSS}

Computing started to close up in the 70's, AT\&T's lockdown of Unix
and Bill Gates' letter to computer hobbyists (Jan'76). Stallman's
story with printer drivers, a reaction to centralized, locked-down
models: the GNU/FSF movement is born. Linux: OSS for the masses. The
rise of the internet as powered by linux.


\subsubsection{Why OSS is relevant to science}
\begin{itemize}
\item Science becomes computing
\item Both building a hierarchical structure
\item freedom to believe differently
\end{itemize}




\subsection{Best practices}


\subsection{Distributed version control and collaboration}



\section{\label{sec:lifecycle}The lifecycle of computational research}

Scientific research has become pervasively computational. In addition
to experiment and theory, the notions of simulation and data-intensive
discovery have emerged as third and fourth pillars of science \cite{4th-paradigm}.
Today, even theory and experiment are computational, as virtually
all experimental work requires computing (whether in data collection,
pre-processing or analysis) and most theoretical work requires symbolic
and numerical support to develop and refine models. Scanning the pages
of any major scientific journal, one is hard-pressed to find a publication
in any discipline that doesn't depend on computing for its findings.

And yet, for all its importance, computing is often treated as an
afterthought both in the training of our scientists and in the conduct
of everyday research. Most working scientists have witnessed how computing
is treated as a task of secondary importance that students and postdocs
learn ``on the go'' with little training to ensure that results
are trustworthy, comprehensible and ultimately a solid foundation
for reproducible outcomes. Software and data are stored with poor
organization, documentation and tests. A patchwork of software tools
is used with limited attention paid to capturing the complex workflows
that emerge, and the evolution of code is often not tracked over time,
making it difficult to understand how a result was obtained. Finally,
many of the software packages used by scientists in research are proprietary
and closed-source, preventing the community from having a complete
understanding of the final scientific results. The consequences of
this cavalier approach are serious. Consider, just to name two widely
publicized cases, the loss of public confidence in the``Climategate''
fiasco \cite{Hef10} or the Duke cancer trials scandal, where sloppy
computational practices likely led to severe health consequences for
several patients \cite{Cou10}. 

This is a large and complex problem that requires changing the educational
process for new scientists, the incentive models for promotions and
rewards, the publication system, and more. We do not aim to tackle
all of these issues here, but our belief is that a central element
of this problem is the nature and quality of the software tools available
for computational work in science. Based on our experience over the
last decade as practicing researchers, educators and software developers,
we propose an integrated approach to computing where the entire life-cycle
of scientific research is considered, from the initial exploration
of ideas and data to the presentation of final results. Briefly, this
life-cycle can be broken down into the following phases:
\begin{itemize}
\item \textbf{Individual exploration:} a single investigator tests an idea,
algorithm or question, likely with a small-scale test data set or
simulation.
\item \textbf{Collaboration:} if the initial exploration appears promising,
more often than not some kind of collaborative effort ensues.
\item \textbf{Production-scale execution:} large data sets and complex simulations
often require the use of clusters, supercomputers or cloud resources
in parallel.
\item \textbf{Publication:} whether as a paper or an internal report for
discussion with colleagues, results need to be presented to others
in a coherent form.
\item \textbf{Education:} ultimately, research results become part of the
corpus of a discipline that is shared with students and colleagues,
thus seeding the next cycle of research.
\end{itemize}
In this project, we tackle the following problem.\textbf{ There are
no software tools capable of spanning the entire lifecycle of computational
research.} The result is that researchers are forced to use a large
number of disjoint software tools in each of these phases in an awkward
workflow that hinders collaboration and reduces efficiency, quality,
robustness and reproducibility.

These can be illustrated with an example: a researcher might use Matlab
for prototyping, develop high-performance code in C, run post-processing
by twiddling controls in a Graphical User Interface (GUI), import
data back into Matlab for generating plots, polish the resulting plots
by hand in Adobe Illustrator, and finally paste the plots into a publication
manuscript or PowerPoint presentation. But what if months later the
researcher realizes there is a problem with the results? What are
the chances they will be able to know what buttons they clicked, to
reproduce the workflow that can generate the updated plots, manuscript
and presentation? What are the chances that other researchers or students
could reproduce these steps to learn the new method or understand
how the result was obtained? How can reviewers validate that the programs
and overall workflow are free of errors? Even if the researcher successfully
documents each program and the entire workflow, they have to carry
an immense cognitive burden just to keep track of everything.


\subsection{The patchwork of existing software tools}

For \textbf{individual exploratory work}, researchers use various
interactive computing environments: Microsoft Excel, Matlab, Mathematica,
Sage \cite{sage}, and more specialized systems like R, SPSS and STATA
for statistics. These environments combine interactive, high-level
programming languages with a rich set of numerical and visualization
libraries. The impact of these environments cannot be overstated;
they are used almost universally by researchers for rapid prototyping,
interactive exploration and data analysis and visualization. However,
these environments have a number of limitations: (a) some of them
are proprietary and/or expensive (Excel, Matlab, Mathematica), (b)
most (except for Sage) are focused on coding in a single, relatively
slow, programming language and (c) most (except for Sage and Mathematica)
do not have a document format that is rich, i.e., that can include
text, equations, images and video in addition to source code. While
the use of proprietary tools isn't a problem \emph{per se} and may
be a good solution in industry, it is a barrier to scientific collaboration
and to the construction of a common scientific heritage. Scientists
can't share work unless all colleagues can purchase the same package,
students are forced to work with black boxes they are legally prevented
from inspecting (spectacularly defeating the very essence of scientific
inquiry), and years down the road we may not be able to reproduce
a result that relied on a proprietary package. Furthermore, because
of their limitations in performance and handling large, complex code
bases, these tools are mostly used for prototyping: researchers eventually
have to switch tools for building production systems.

For \textbf{collaboration}, researchers currently use a mix of email,
version control systems and shared network folders (Dropbox, etc.).
Version control systems (Git, SVN, CVS, etc.) are critically important
in making research collaborative and reproducible. They allow groups
to work collaboratively on documents and track how those documents
evolve over time. Ideally, all aspects of computational research would
be hosted on publicly available version control repositories, such
GitHub or Google Code. Unfortunately, the most common approach is
still for researchers to email documents to each other. This form
of collaboration makes it nearly impossible to track the development
of a large project and establish reproducible and testable workflows.
When it works at all, it most certainly doesn't scale beyond a very
small group, as painfully experienced by anyone who has participated
in the madness of a flurry of email attachments. 

For \textbf{production-scale execution}, researchers are forced to
turn away from the convenient interactive computing environments to
compiled code (C/C++/Fortran) and parallel computing libraries (MPI,
Hadoop), as most interactive systems don't provide the performance
necessary for large-scale work and have primitive parallel support.
These tools are difficult to learn and use and require large time
investments. We emphasize that before production-scale computations
begin, the researchers have already developed a mostly functional
prototype in an interactive computing environment. Turning to C/C++/Fortran
for production means starting over from scratch and maintaining at
least two versions of the code moving forward. Furthermore, data produced
by the compiled version has to be imported back into the interactive
environment for visualization and analysis. The resulting back-and-forth,
complex workflow is nearly impossible to capture and put into version
control systems, again making the computational research difficult
to reproduce.

For \textbf{publications} and\textbf{ presentations}, researchers
use tools such as \LaTeX{}, Google Docs or Microsoft Word/PowerPoint.
The most important attribute of these tools in this context is that
they don't integrate well with version control systems (\LaTeX{} excepted)
and with other computational tools. Digital artifacts (code, data
and visualizations) have to be manually pasted into these documents,
so that the same content is duplicated in many different places. When
the artifacts change, the documents quickly become out of sync. 


\subsection{IPython}

We propose that the open source IPython project \cite{PER-GRA:2007}
offers a solution to these problems; a single software tool capable
of spanning the entire life-cycle of computational research. Amongst
high-level open source programming languages, Python is today the
leading tool for general-purpose source scientific computing (along
with R for statistics), finding wide adoption across research disciplines,
education and industry and being a core infrastructure tool at institutions
such as CERN and the Hubble Space Telescope Science Institute \cite{Perez2011,ganga09,SST}.
The PIs created IPython as a system for interactive and parallel computing
that is the\emph{ de facto} environment for scientific Python. In
the last year we have developed the IPython Notebook, a web-based\emph{
interactive computational notebook} that combines code, text, mathematics,
plots and rich media into a single document format (see Fig.~\ref{fig:IPython-notebook}).
The IPython Notebook was designed to enable researchers to move fluidly
between all the phases of the research life-cycle and has gained rapid
adoption. It provides an integrated environment for all computation,
without locking scientists into a specific tool or format: Notebooks
can always be exported into regular scripts and IPython supports the
execution of code in other languages such as R, Octave, bash, etc.
In this project we will expand its capabilities and relevance in the
following phases of the research cycle: interactive exploration, collaboration
and publication/presentation.

\begin{figure}
\begin{centering}
\includegraphics[width=3.2in]{fig/ipython-notebook-specgram.png}
\par\end{centering}

\caption{\label{fig:IPython-notebook}The web-based IPython Notebook combines
explanatory text, mathematics, multimedia, code and the results from
executing the code.}
\end{figure}



\section{Conclusion}\label{conclusion}

\subsection{Changing the culture of science and the role of incentive models}

Science has become computational, but the incentive models of science
are single-mindedly focused on paper-oriented publications that completely
ignore the very existence of a computational process. Papers are accepted 

Open{*}: software, access (Elsevier), education, review.


\bibliographystyle{plain}
\bibliography{ipython}

\end{document}
